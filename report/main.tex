\documentclass[a4paper]{article}
\usepackage[latin1]{inputenc}
\usepackage{geometry}
\usepackage{amssymb}
\usepackage{framed}
\usepackage{amsmath}
\usepackage{graphicx}
\usepackage{booktabs}
\usepackage{subcaption}

\setlength{\parindent}{0pt}
\setlength{\parskip}{3ex}

\begin{document}

\begin{center}
  {\large Artificial Neural Networks and Deep Architectures, DD2437}\\
  \vspace{7mm}
  {\huge Short report on lab assignment 4\\[1ex]}
  {\Large Restricted Boltzmann Machines and Deep Belief Networks}\\
  \vspace{8mm}  
  {\Large Hilding Wollbo\\}
  \vspace{4mm}
  {\large September 7, 2020\\}
\end{center}

%\begin{framed}
%Please be aware of the constraints for this document. The main intention here is that you learn how to select and organise the most relevant information into a concise and coherent report. The upper limit for the number of pages is 6 with fonts and margins comparable to those in this template and no appendices are allowed. \\
%These short reports should be submitted to Canvas by the authors as a team before the lab presentation is made. To claim bonus points the authors should uploaded their short report a day before the bonus point deadline. The report can serve as a support for your lab presentation, though you may put emphasis on different aspects in your oral demonstration in the lab.
%Below you find some extra instructions in italics. Please remove them and use normal font for your text.
%\end{framed}

\section{Main objectives and scope of the assignment}

\section{Method}
The tasks were implemented in Python using numpy and pandas for data manipulation while matplotlib was used to produce the graphics in the report.
\section{Restricted Boltzmann Machine}

\section{Deep Belief Network}

\subsection{Pre-training}

\subsection{Fine-tuning}

\section{Final remarks}

\end{document}